\section{May 2, 2019}

\begin{theorem}
    Let $f$ be analytic and nonvanishing in a simply connected
    region $U$.
    Then there exists a \textit{single-valued analytic branch} of
    $\log f(z)$:  an analytic funciton $g(z)$ defined
    on $U$ such that 
    \[ e^{g(z)} \equiv f(z) \]
    for all $z \in U$.
\end{theorem}

\begin{proof}
    Consider $f'(z)/f(z)$.
    Note that it is analytic in $U$ because $f$ is nonvanishing.
    By Cauchy Goursat we know $f'(z)/f(z)$ has an antiderivative,
    call it $F(z)$.
    Consider $f(z)e^{-F(z)}$.
    Note that
    \[ \frac{d}{dz} [f(z)e^{-F(z)}] = f'(z)e^{-F(z)} - f(z)F'(z)e^{-F(z)}
    = f'(z)e^{-F(z)} - f'(z) e^{-F(z)} \equiv 0. \]
    This means we can pick a $z_0$ such that $f(z)e^{-F(z)}=
    f(z_0)e^{-F(z_0)}$.
    We can pick a branch of log and solve for $f(z)$ to find:
    \[ f(z) = e^{\Log f(z_0) + F(z) - F(z_0)}, \]
    so we can let $g(z) = \Log f(z_0) + F(z) - F(z_0)$ as
    desired.
\end{proof}

\subsection{Further Versions of Cauchy Goursat}
\begin{theorem}
    Let $f$ be analytic in \textit{region} $U$.
    Suppose $\sigma$ is a closed \textit{zig-zag} polygon path
    inside $U$ such that $n(\sigma, a) = 0$ for all $a \notin U$.
    Then
    \[ \int_{\sigma} f = 0 \]
\end{theorem}

Note how the $n(\sigma, a) = 0$ condition replaces the simple
connectivity of $U$.
Also note that this theorem is not sufficient for $f$ to be
independent of path.
But the proof is still identical because none of the necessary
conditions were taken away.

\begin{exercise}
    Prove the above theorem, except $\sigma$ is \textit{any} path
    inside $U$ such that $n(\sigma, a) = 0$.
\end{exercise}

\subsection{Homology Form of Cauchy Goursat}
\begin{definition}
    We say that a cycle $\beta$ in region $U$ is \term{homologous
    to $0$ modulo $U$} written
    \[ \beta \sim 0 \pmod{U}, \]
    if for all $a \in U$, $n(\beta, a) = 0$.
    We also say $\beta$ ``bounds in $U$.''
\end{definition}

\begin{definition}
    Two cycles $\beta_1$ and $\beta_2$ are \term{homologous 
    modulo $U$}, denoted $\beta_1 \sim \beta_2$ if
    \[ [\beta_1] - [\beta_2] \sim 0 \pmod{U}. \]
\end{definition}

\begin{theorem}[Homology Form of Cauchy Goursat]
    If $\beta$ is a cycle inside $U$ such that $\beta \sim 0
    \pmod{U}$ and if $f$ is analytic on $U$, then
    \[ \int_{\beta} f = 0. \]
\end{theorem}

The outline of the proof is to cut each $\gamma_j$ in the formal sum
of $\beta$ into a number of pieces and close each piece,
say $\gamma_{jk}$, with an L-shaped path $\lambda_{jk}$.
By the original Cauchy Goursat each of those loops give a zero integral,
so it remains to show there are a finite number of $\gamma_{jk}$'s.
There are finitely many because there exists a distance between two
closed sets, one of which is compact.

\begin{definition}
    The \term{connectivity} of a region $U$ is the number of compact
    components in $\C \setminus U$.
\end{definition}

We're going to assume $U$ has finite connectivity $n$ and write it as:
\[ \C \setminus U = A_0 \cup \underbrace{A_1 \cup A_2 \cup \cdots A_n}_{\text{compact}}, \]
where $\infty \in A_0$.

\begin{theorem}
    Any cycle $\beta$ in $U$ determines unique integer
    coefficients $k_1, \ldots, k_n$ such that
    \[ \beta \sim k_1[\gamma_1] + \cdots k_n[\gamma_n]
    \pmod{U}. \]
\end{theorem}

\begin{definition}
    The $j$th \term{period} of $f$ is defined as
    \[ P_j(f) := \int_{\gamma_j} f. \]
\end{definition}
Notice that the $P_j$ does not depend on $\beta$.
This is because if we have another loop $\eta_j$ that winds around the same hole the
same number of times, we can connect them with a segment and make that
a loop.  This implies $\int_{\gamma_j} = \int_{\eta_j}$.

We also sometimes write $P(f, a) := P_j(f)$ if $a \in A_j$.

\begin{problem*}
    Prove there exists an analytic branch of $\sqrt{1-z^2}$ defined
    in any region $U$ such that $1, -1$ lie in the same component
    of $\hat{\C} \setminus U$.
\end{problem*}

Here is a key property: an analytic function $f$ has an antiderivative $F$
in $U$ if and only if $f$'s periods with respect to $U$ are all $0$.

\begin{theorem}
    If $U$ is a region of finite connectivity $n$, then there exists
    a \term{homology basis} $(\gamma_1, \ldots, \gamma_n)$ for $U$,
    namely an ordered list of cycles inside $U$ such that if the
    bounded components of $\C \setminus U$ are $A_1, \ldots, A_n$,
    then
    \[ n(\gamma_j, A_k) = \delta_{jk}, \]
    where $\delta_{jk}$ is the \textit{Kronecker Delta Function}.
\end{theorem}

\begin{theorem}
    Any cycle $\beta$ in $U$ is homologous to a unique $\Z$-linear
    combination
    \[ \beta \sim n(\beta, A_1)\beta_1 + n(\beta, A_2)\beta_2 + \cdots
    + n(\beta, A_n)\beta_n. \]
\end{theorem}

\subsection{Cauchy Integral Formula Revisited}
\begin{theorem}
    Let $f$ be analytic in $U$ and $\beta$ be a cycle such that $\beta
    \sim 0 \pmod{U}$ and $a \notin \ran \beta$.
    Then
    \[ n(\beta, a)f(a) = \frac{1}{2\pi i} \int_{\beta} \frac{f(w)}{w-a}
    \; dw. \]
\end{theorem}

\begin{proof}
    The proof is virtually the same as the first one, only that we use
    the more general Cauchy Goursat Theorem on cycles.
\end{proof}

\subsection{The Residue Theorem}
Let $a$ be a complex number.
\begin{definition}
    Let $a$ be a complex number in $U$, and let $D_{\delta}(a) \subseteq U$.
    The \term{residue} of $f$ at $z=a$ is the unique number $R$ such that
    $f(z) - \frac{R}{z-a}$ has an antiderivative in the punctured disk
    $0 < |z-a| < \delta$.

    We write $\Res_{z=a} f(z) := R$ and $\Res_a f := R$ too.
\end{definition}
We can compute what the residue is in terms of the period:
\[ 0 = P(f - R/(z-a)) = \oint_{C_{\delta/2}(a)} f(z) - \frac{R}{z-a} \; dz \]
Rearranging gives
\[ R \oint_{C_{\delta/2}(a)} \frac{1}{z-a} \; dz = P(f, a) \]
Therefore,
\[ R = \frac{1}{2\pi i} P(f, a). \]

Let's see how we can compute the residue in the case where $a$ is a pole.
Write $f(z) = b_n{(z-a)}^{-n} + b_{n-1}{(z-a)}^{-n+1} + \cdots + b_2{(z-a)}^{-2}
+ b_1{(z-a)}^{-1} + \phi(z)$, where $\phi$ is analytic in $D_{\delta}(a)$.
Note that $b_j{(z-a)}^{-j}$ has an antiderivative in $D_{\delta}(a)$ for
$j \ge 2$, namely $b_j/(-j+1) {(z-a)}^{-j+1}$.
Also, since $\phi$ is analytic, $P(\phi, a) = 0$.
Therefore, $R = b_1$.

From this it follows that if $\ord(f, a) = -1$, then
$\Res_a f = \lim_{z \to a} f(z) \cdot (z-a)$.
Here's the more general theorem.
\begin{theorem}
    If $a$ is a pole of $f$ of order $-n$, then
    \[ \Res_{z=a} f(z) = \frac{1}{(n-1)!} \lim_{z \to a} \frac{d^{n-1}}{dz^{n-1}} [{(z-a)}^n f(z)]. \]
\end{theorem}
The proof just requires you to differentiate the \textit{Laurent Expansion}
of $f$ exactly $n-1$ times and grabbing the residue of that function.

\begin{exercise*}
    If $f(z) = \frac{e^z}{(z-1)(z-2i)}$, compute $\Res_1 f$ and $\Res_{2i} f$.
\end{exercise*}

\begin{exercise*}
    If $g(z) = \frac{e^z}{{(z-1)}^2}$, compute $\Res_1 g$.
\end{exercise*}

\begin{theorem}[Residue Theorem]
    Let $\beta$ be a cycle in $U$ and let the bounded components
    of $\C \setminus U$ be $A_1, \ldots, A_n$. 
    Pick $a_j \in A_j$.
    Then
    \[ \frac{1}{2\pi i} \int_{\beta} f = \sum_{j=1}^{n} n(\beta, a_j)
    \Res_{z=a_j} f(z). \]
\end{theorem}

