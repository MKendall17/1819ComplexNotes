\section{March 22, 2019}

\subsection{Analytic Case of the Argument Principle}

\begin{theorem}[Argument Principle]
    Let $f$ be analytic in an open disk $\Delta$ and let the
    roots of $f$ in $\Delta$ be $a_1, a_2, \ldots$ listed
    with multiplicities.
    For some subset $S$ of $\Delta$, let
    \[ Z(f, S) = \sum_{j \, : \, a_j \in S} \ord (f, a_j). \]
    Then for any counterclockwise oriented circle $C \subseteq
    \Delta$,

    \[ Z(f, \inter{C}) = \frac{1}{2\pi i} \oint_C \frac{f'(z)}{f(z)} \, dz. \]
\end{theorem}

\begin{proof}
    We split this proof into two cases.
    \begin{enumerate}
        \item There are a finite number of roots of $f$:
            $a_1, a_2, \ldots, a_n$.
            We can then factor $f$ as
            \[ f(z) = (z-a_1)(z-a_2) \cdots (z-a_n)g(z), \]
            for some analytic $g$ that is nonzero in $\Delta$.
            Now note
            \[ \frac{f'(z)}{f(z)} = \frac{1}{z - a_1} + 
                \frac{1}{z-a_2} + \cdots + \frac{1}{z-a_n} + 
                \frac{g'(z)}{g(z)}.
            \]
            By Cauchy-Goursat, note that
            \[ \oint_C \frac{g'(z)}{g(z)} \, dz = 0.\]
            Therefore
            \[ \frac{1}{2\pi i} \oint_C \frac{f'(z)}{f(z)} \, dz
                = n(C, a_1) + n(C, a_2) + \cdots + n(C, a_n) 
                + 0
                =\sum_{j \, : \, a_j \in S} \ord (f, a_j) ,
            \]
            as desired.

        \item There are an infinite number of roots of $f$:
            $a_1, a_2, \ldots$.
            Take a slightly smaller disk $\Delta'$ so that
            $\ran \gamma \subseteq C$.
            Note that there cannot be an infinite
            number of roots in $\Delta'$ because any
            accumulation point of a subsequence of ${(a_k)}_{k=1}^{\infty}$
            must be on the boundary of $\Delta'$ which is inside
            $\Delta$.
            Therefore, we can repeat the same argument as above
            for the finite case on $\Delta'$.
    \end{enumerate}
\end{proof}

\begin{theorem}[Generalized Argument Principle]
    Let $f$ be analytic in an open disk $\Delta$ and let 
    the roots of $f$ be listed $a_1, a_2, \ldots$ with
    multiplicities.
    Let $\gamma$ be a piecewise smooth loop inside $\Delta$.
    Then
    \[
        n(f \circ \gamma, 0)
        = \sum_{f(a) = 0} \ord (f, a) n(\gamma, a)
        = \frac{1}{2\pi i} \oint_{\gamma} \frac{f'(z)}{f(z)} \,
        dz.
    \]
\end{theorem}

\subsection{Change of Variables}

\begin{theorem}
    Let $\gamma$ be a piecewise smooth loop path $\gamma: [0,1]
    \to \C$, $g$ be analytic in some region $U$ containing
    $\ran \gamma$, and
    $f$ be a continuous function on $\ran (g \circ \gamma)$.
    Then
    \[
        \int_{\gamma} f(g(z)) g'(z) \, dz = 
        \int_{g \circ \gamma} f(w) \, dw.
    \]
\end{theorem}

\begin{proof}
    By the chain rule,
    \begin{align*}
        \int_{\gamma} f(g(z)) g'(z) \, dz &= 
        \int_0^1 f(g(\gamma(t))) g'(\gamma(t)) \gamma'(t) \, dt\\
        &= \int_0^1 f(g(\gamma(t))) \frac{d}{dt} (g(\gamma(t)))
        \, dt \\
        &= \int_{g \circ \gamma} f(w) \, dw.
    \end{align*}
\end{proof}

\subsection{Local Correspondance Theorem}

\begin{theorem}[LCT]
    Let $f$ be analytic in region $U$, let $z_0 \in U$, $w_0 = f(z_0)$, 
    and
    \[ n = \ord(f-w_0, z_0). \]
    Then there exists $\varepsilon_0 > 0$ such that for all
    $0 < \varepsilon \le \varepsilon_0$, there exists a $\delta_0 > 0$
    such that for all $w \in D_{\delta} (w_0)$ there exist $n$
    solutions (counting multiplicities) to the equation $f(z) = w$
    lying in $D_{\varepsilon}(z_0)$.
\end{theorem}

\begin{proof}
    Choose $\varepsilon_0$ small enough so that $D_{\varepsilon_0}
    (z_0) \subseteq U$ and $z_0$ is the only root of $f$ in that 
    disk.
    Let $C = C_{\varepsilon_0}(z_0)$ and $\Gamma = f \circ C$.
    Now a simple lemma.
    \begin{lemma}
        For any $a, p \in \C$,
        \[ n(\gamma, p) = n(\gamma + a, p + a). \]
    \end{lemma}

    \noindent
    The proof of this lemma is just an application of the chain
    rule.
    Now note that
    \[ n(\Gamma, w_0) = n(\Gamma - w_0, 0) = Z(f-w_0, \inter(C)) 
    = n. \]
    Since $w_0$ and $\Gamma$ are open sets, we can choose a $\delta$
    such that $D_{\delta} (w_0)$ is in the same connected component
    as $w_0$.
    Choose a $w_1 \in D_{\delta}(w_0)$.
    We know
    \[ n(\Gamma, w_1) = n(\Gamma, w_0) = n. \]
    Thus, for every $w_1 \in D_{\delta}(w_0)$ there are $n$ points
    locally around $z_0$ that map to it.
\end{proof}

A corollary of the local correspondance theorem is that $f$ is 
an \term{open mapping}: for an open set $V$, $f(V) = \{f(z) \; | \; 
z \in V\}$ is open. 
Take a point $w_0 = f(z_0) \in f(V)$.
By the Local Correspondance Theorem,
there exists a sufficiently small $\varepsilon$ such that 
the disk $D_{\varepsilon}(z_0)$ is fully 
contained in $V$ and a $\delta$ such that $D_{\delta}(w_0) \subseteq f(V)$ and 
all $w \in D_{\delta}(w_0)$ have preimages in $D_{\varepsilon}(z_0)$.
Since there is an open disk around every point in $f(V)$
contained in $f(V)$, $f(V)$ is open.

\subsection{Maximum Principle}

\begin{theorem}[Maximum Principle]
    Let $f$ be analytic and nonconstant in region $U$.
    Let $V$ be any open set inside $U$.
    Then $|f(z)|$ cannot take a maximum for $z$ inside $V$.
\end{theorem}

\begin{proof}
    Suppose for sake of contradiction there exists a $z_0$ 
    such that $|f(z_0)|$ is the maximum of $|f(z)|$ on $V$.
    Because $f$ is an open map, 
    there exists a disk around $f(z_0)$ contained in
    $f(V)$.  But this means we can move away from the origin in
    that disk, gaining a larger modulus, which is a 
    contradiction.
\end{proof}

\begin{cor}
    If $K$ is a compact subset of $U$, then there exists a
    $z_0 \in \bd(K)$ such that
    \[ |f(z)| < |f(z_0)| \]
    for all $z \in \inter(K)$.
\end{cor}

\subsubsection{Other formulations of the Maximum Principle}
\begin{itemize}
    \item Negative formulation:
        \begin{itemize}
            \item If $f$ is analytic in an open set $V$ and achieves
                a maximum $z_0 \in V$, then $f$ is constant.
        \end{itemize}

    \item Positive formulations:
        \begin{itemize}
            \item If $f$ is analytic and nonconstant in some
                open superset of a compact set $K$, then
                $|f(z)|$ achieves a maximum on the boundary of $K$,
                and every point in the interior of $K$ has a
                smaller modulus.
            \item If $f$ is analytic in some open superset of a
                compact set $K$ and if $|f(z)|$ attains a 
                maximum value at some $z_0 \in \inter K$,
                then $f$ is constant.
        \end{itemize}
\end{itemize}

\subsection{Schwartz's Lemma}

\begin{lemma}
    Let $f$ be analytic in the open disk $D_1(0)$ and 
    $f : D_1(0) \to \overline{D_1(0)}$.
    Also suppose $f(0)=0$.
    Then
    \begin{enumerate}
        \item $|f(z)| \le |z|$.
        \item $|f'(0)| \le 1$.
        \item If equality holds in either $1$ or $2$ for any $z$,
            then $f$ is a pure rotation.
    \end{enumerate}
\end{lemma}

\begin{proof}
    Let
    \[
        g(z) = \begin{cases}
            \frac{f(z)}{z} & z \ne 0, \\
            f'(0) & z = 0.
        \end{cases}
    \]
    First of all, $g$ is analytic at $z \ne 0$ because 
    $f$ and $1/z$ are analytic at $z \ne 0$.
    Note that $g$ is also analytic at $z=0$ because 
    \[ \lim_{z \to 0} \frac{f(z) - 0}{z - 0} = f'(0). \]
    We claim $|g(z)| \le 1$ for all $z \in D_1(0)$.
    Suppose there is a $z_0$ and $\varepsilon_0 > 0$
    such that $|g(z_0)| = 1 + \varepsilon_0$.
    Select an $r$ such that $r > |z_0|$ and $r > \frac{1}{1+\varepsilon_0}$.
    Then for $|z| \ge r$,
    \[ |g(z)| = \frac{|f(z)|}{|z|} \le \frac{1}{r} < 1 + 
    \varepsilon_0, \]
    a contradiction.
    Therefore, $|g(z)| \le 1$ so $|f(z)| \le |z|$ on $z \in D_1(0)$
    (remember $f(0)=0$). 
    This proves part $1$.
    Since $|g(0)| \le 1$, $|f'(0)| \le 1$, which proves part $2$.
    Finally, if equality holds holds in either part $1$ or $2$,
    $g$ takes on a maximum, which means
    it has constant modulus.
    From $f \equiv c$ section, $g$ must be constant, so
    $f$ is a pure rotation.

\end{proof}
