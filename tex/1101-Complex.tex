\section{Nov. 1, 2018}
\subsection{Real Valued Function of a Complex Variable}

\begin{theorem}
    If $f:D \subseteq \C \to \R$ and if $f'(z)$ exists for all $z \in \inter{D}$ where $D$ is connected, then $f(z) \equiv c \in \mathbb{R}$ for all $z \in \inter{D}$.
\end{theorem}

\begin{proof}
First, a lemma.
    \begin{lemma}
        If $f'(z)$ exists, then $f'(z) = 0$.
    \end{lemma}

    \begin{proof}
        Let $h \to 0$ and $h \in \mathbb{R}$.
        In the difference quotient $\frac{f(z+h)-f(z)}{h}$, 
        $f(z+h) - f(z)$ is real because $f$ is real valued, 
        and $h$ is real by choice.
        Therefore, the $f'(z) := \lim_{h \to 0} \frac{f(z+h) - f(z)}{h}$ is real.

        Now let $h \to 0$ and $h \in i\R$.  
        The difference quotient is now imaginary because the numerator is still real and the denominator is imaginary.
        Therefore, $f'(z)$ is imaginary.

        Since $f'(z)$ is both real and imaginary, $f'(z) = 0$.
    \end{proof}

    Now choose $p,q \in D^0 := {\text Int}\; D$.
    Let $\gamma(t)$ be a path from $p$ to $q$ ($\gamma(0) = p$, $\gamma(1) =q$ and $\gamma$ is piecewise-continuous).
    
    Let $g(t) := f(\gamma(t))$.
    Note that
    \begin{align*}
        g'(t) = \lim_{\Delta t \to 0} \frac{f(\gamma(t+\Delta t) - f(\gamma(t)))}{\gamma(t + \Delta t) - \gamma(t)}\frac{\gamma(t + \Delta t) - \gamma(t)}{(t + \Delta t) - t} 
    \end{align*}
    We know that $\gamma$ is continuous (?) at $t$, so $\gamma(t+\Delta t) - \gamma(t)$ can get arbitrarily small.

    This means:
    \[\lim_{\Delta t \to 0} \frac{f(\gamma(t+\Delta t) - f(\gamma(t)))}{\gamma(t + \Delta t) - \gamma(t)}\ = f'(\gamma(t)) \]
    The other difference quotient is:

    \[ \frac{\gamma(t + \Delta t) - \gamma(t)}{(t + \Delta t) - t}
    = \frac{(q-p)\Delta t}{\Delta t} = (q-p) \]

    Putting this all together,
    \begin{align*}
        g'(t) &= \lim_{\Delta t \to 0} \frac{f(\gamma(t+\Delta t) - f(\gamma(t)))}{\gamma(t + \Delta t) - \gamma(t)}\frac{\gamma(t + \Delta t) - \gamma(t)}{(t + \Delta t) - t} \\
        &= f'(\gamma(t))\cdot(q-p) \equiv 0.
    \end{align*}

    From MVT\footnote{Suppose for sake of contradiction there exist $p,q$ such that $g(p) \ne g(q)$.  Then by MVT there exists a $c \in (p,q)$ such that $(g(q)-g(p))/(q-p) \ne 0$.  This is a contradiction because $g(t) \equiv 0$.}, this means $g(t) = c \in \R$.
    
    Finally, we get
    \[ f(p) = f(\gamma(0)) = g(0) = g(1) = f(\gamma(1)) = f(q). \]

    We can extend this argument by finding a polygon path from $p \to q$ for any $p,q \in D$.
    We can do this because $D$ is compact.




\end{proof}

