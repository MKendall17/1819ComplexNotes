\section{Sept. 22, 2018}

\begin{definition}
    $\cis{\theta} := {(-1)}^{\theta / \pi}$, where
    \[ \pi := \lim_{n \to \infty} 2^n \sqrt{2 - \sqrt{2 + \square}^{O(n-1)} (0)}. \]
    For example, the first couple terms that approach $\pi$ are $4\sqrt{2 - \sqrt{2}}$, $8\sqrt{2 - \sqrt{2 + \sqrt{2}}}$, and $16\sqrt{2 - \sqrt{2 + \sqrt{2 + \sqrt{2}}}}$.

    These terms approximate the perimeter of the top half of a semicircle with $2^n$ chords.
\end{definition}

\begin{definition}
    We define cosine and sine:
    \begin{align*}
        \cos(\theta) := \real{\cis{\theta}} \\
        \sin(\theta) := \imag{\cis{\theta}}
    \end{align*}
\end{definition}

We can show some properties of cis, such as
\begin{itemize}
    \item$ \cis{-\theta} = \overline{\cis{\theta}}$
    \item $\cis{\theta + \phi} = {(-1)}^{\theta + \phi} = \cis{\theta}\cis{\phi}$
\end{itemize}

\subsection{Square roots in $\C$}
\begin{theorem}
    A complex number $z$ cannot have more than $2$ square roots.
\end{theorem}
\begin{proof}
    Write $z = w^2 = v^2$ for some not necessarily distinct $w$ and $v$.
    Then we see $w^2 - v^2 = (w-v)(w+v) = 0$ which means $w = v$ or $w = -v$.
    Therefore, $z$ can have at most two square roots.
\end{proof}

\begin{theorem}
    The square root of a complex number $z = a + bi$ can be the following:
    \[ \sqrt{z} := \begin{cases}
    0 & \text{if $a=b=0$}, \\
    \pm \sqrt{a} & \text{if $a>0$ and $b=0$}, \\
    \pm i\sqrt{a} & \text{if $a<0$ and $b=0$}, \\
    \pm \frac{1}{\sqrt{2}}\left(\sqrt{\sqrt{a^2 + b^2} + a} 
    + i\, \sgn{b} \sqrt{\sqrt{a^2+b^2}-a} \right) & \text{otherwise}.
    \end{cases} \]
\end{theorem}

\begin{definition}
    We define the \term{principal square root} of $z = a+bi$ to be:
    \begin{itemize}
        \item $0$ if $a=b=0$
        \item $\sqrt{a}$ if $a>0$ and $b = 0$
        \item $i\sqrt{a}$ if $a < 0$ and $b = 0$.
        \item $\frac{1}{\sqrt{2}}\left( \sqrt{ \sqrt{a^2 + b^2} + a} 
            + i\, \sgn{b} \sqrt{\sqrt{a^2+b^2}-a} \right)$ otherwise. 
    \end{itemize}
\end{definition}


\subsection{Defining $(-1)^t$}

\begin{definition}
    Define the set of \term{diadic rationals} $D$ to be
    \[ D = \{ \frac{k}{2^n} \, | \, k \in \Z, n \in \N \}. \]
\end{definition}

We will first define $(-1)^d$ where $d \in D$.
Let $d = \frac{k}{2^n}$ for some $k \in \Z$ and $n \in \N$.
Our first step is to create a sequence $p_n$ defined as follows:
\[ p_n = a_n + i \sqrt{1 - a_n^2}, \]
and 
\[ p_{n+1} = \sqrt{p_n}. \]
We let $a_n$ be defined by:
\[ a_n = \begin{cases}
        -1 & \text{if $n = 1$,} \\
        a_n = \sqrt{\frac{1 + a_n}{2}} & \text{if $n > 1$.}
    \end{cases}
\]
Now let
\[ (-1)^d = (-1)^{\frac{k}{2^n}} := p_n^k. \]
We can now define $(-1)^t$ for $t \in \R$:
\[ (-1)^t := \lim_{n \to \infty} (-1)^{\frac{\floor{2^nt}}{2^n}}
    = \lim_{n \to \infty} p_n^{\floor{2^nt}}.
\]
Why does this limit exist?

\noindent
To answer that question, we need to take a tangent.

\subsubsection{Wedge Products}

\begin{definition}
    Let the \term{wedge product} of $z$ and $w$ be
    \[ z \wedge w = \imag (\overline{z}w). \]
\end{definition}

Why is this useful?

\begin{theorem}
    If $z$ and $w$ have positive imaginary parts and 
    $z \wedge w > 0$, then $w$ is counterclockwise from $z$.
\end{theorem}

\begin{proof}
    Let $z = a + bi$ and $w = c + di$.
    Then 
    \[ z \wedge w = \imag(\overline{z}w) = ad - bc. \]
    But note that
    \[ ad - bc = \det \begin{bmatrix} 
                 a & c \\
                 b & d
            \end{bmatrix}
            > 0.
         \]
    This means the vector $\begin{bmatrix} c \\ d \end{bmatrix}$ 
    is counterclockwise from $\begin{bmatrix} a \\ b \end{bmatrix}$, which is what we wanted to prove.
\end{proof}

Now the other useful piece.

\begin{theorem}
    If $z, w$ are on the upper semicircle ($\mathbb{S}^+$),
    then $z \wedge w > 0$ if and only if $\real z > \real w$.
\end{theorem}

\begin{proof}
    Let $\imag z = x$ and $\imag w = y$.
    We wish to show $z \wedge w > 0$ iff $x > y$.
    
    Note that $z = x + i\sqrt{1 - x^2}$ and 
    $w = y + i\sqrt{1 - y^2}$.
    Then we want to show
    \[ \imag(\overline{z}w) = x\sqrt{1 - y^2} - y\sqrt{1 - x^2}
    > 0 \]
    if and only if $x > y$.
    Let's first prove that if $x > y$, then $z \wedge w > 0$.
    There are three cases to consider.

    \begin{enumerate}
        \item $0 \ge y < x$.
            Then
            \begin{align*}
                x^2 &> y^2 \\
                x^2 - x^2y^2 &> y^2 - x^2y^2 \\
                \frac{x^2}{y^2} &> \frac{1 - x^2}{1 - y^2} \\
                \frac{x}{y} &> \frac{\sqrt{1 - x^2}}{\sqrt{1- y^2}} \tag{$x$ and $y$ are positive} \\
                x\sqrt{1-y^2} - y\sqrt{1-x^2} &> 0 \\
                \imag{\overline{z}w} &> 0,
            \end{align*}
            as desired.

        \item $y < 0 \ge x$.
            This case is simple because $-y\sqrt{1-x^2} > 0$, 
            so $\imag{\overline{z}w} > 0$.

        \item $y < x < 0$.
            Note that this means $0 < -x < -y$, and by
            the first case, this means 
            \[ (-y)\sqrt{1 - (-x)^2} - (-x)\sqrt{1 - (-y)^2}
            > 0. \]
            This is equivalent to 
            \[ \imag{\overline{z}w} = x\sqrt{1 - y^2} - 
            y\sqrt{1 - x^2} > 0, \]
            as needed.
    \end{enumerate}

    By reversability of the previous argument, we can prove
    the converse: if $z \wedge w > 0$, then $\real{z} > \real{w}$.

\end{proof}

Now to the next step.

% define wedge product
% show w is ccw from z iff z wedge w > 0 (z, w in upper half)
% show z wedge w > 0 iff y < x (for z and w on upper semicircle)

\subsubsection{Floors}
\begin{theorem}
    For any $x \in \R$,
    \[ \floor{2x} \le 2\floor{x}. \]
\end{theorem}

\begin{proof}
    Note that for any $x$,
    \[ \floor{x} \le x < \floor{x} + 1. \]
    This means 
    \[ 2\floor{x} \le 2x < 2\floor{x} + 2. \]
    This means $\floor{2x} < 2\floor{x}$.
\end{proof}

Now we're ready.

\subsubsection{Existence of Limit}
% easy limit thing with induction



% p_n+1 is ccw p_n and the x coords have lower limit
% make sense of this later.

Note that $p_{n+1}$ is counterclockwise from $p_n$ because
\[ \imag{\overline{p_n}p_{n+1}} = \imag{\overline{p_n}
p_np} = \imag{p} > 0. \]


% show floor 2x < 2 floor x

\begin{enumerate}
    \item $p_n = x_n + iy_n$ converges as $n \to \infty$
        because $x_n$ is monotonically increasing sequence
        and $y_n$ is a continuous function of $x_n$.

    \item $\frac{\floor{2^nt}}{2^n}$ monotonically increases
        to $t$ as $n \to \infty$.
\end{enumerate}

If I let $q_k = p_k^{\floor{2^kt}}$, I can show $q_k$ is
a Cauchy sequence?

%Not sure why these imply the limit exists (?).

\subsection{Consistency of Diadic and Limit Definitions}

Let $t = d = \frac{k}{2^n}$.
We wish to show $(-1)^t = (-1)^d$.
Pick sufficiently large $n$ such that $\floor{2^n t}$ is
an integer.
\begin{align*}
    (-1)^t &= \lim_{n \to \infty} p_n^{\floor{2^n}t} \\
    &= \lim_{n \to \infty} \left( p_n^{2^{n-N}} \right)^k 
    \tag{$n$ is large enough}\\
    &= \lim_{n \to \infty} p_N^k 
    \tag{by recursion of $p_n$}\\
    &= p_N^k \\
    &= (-1)^d.
\end{align*}

\subsection{Properties of $(-1)^t$}

We'd expect some exponent properties to hold.

\begin{theorem}
    For real numbers $t,s$, ${(-1)}^{t+s} = {(-1)}^{t}\cdot {(-1)}^{s}$.
\end{theorem}

\begin{proof}
    Note that $\floor{2^n(t+s)} = \floor{2^n t} + \floor{2^n s} + \epsilon_n$, 
    where $\epsilon_n \in \{0,1\}$.
    Now,
    \[ {(-1)}^{t+s} = \lim_{n \to \infty} p_n^{\floor{2^n(t+s)}} = \lim_{n \to \infty} p_n^{\floor{2^n t}} p_n^{\floor{2^n s}} p_n^{\epsilon_n} \]
    Since $|p_n^{\epsilon_n} -1| \to 0$, we see that
    \[ {(-1)}^{t+s} = \lim_{n \to \infty} p_n^{\floor{2^n t}} p_n^{\floor{2^n s}} p_n^{\epsilon_n} = \lim_{n \to \infty} p_n^{\floor{2^n t}}  \lim_{n \to \infty} p_n^{\floor{2^n s}} \lim_{n \to \infty} p_n^{\epsilon_n} = {(-1)}^{t}\cdot {(-1)}^{s}, \]
    as desired.
\end{proof}

\begin{theorem}
    If $(-1)^t = (-1)^s$, then $t \equiv s \pmod{2}$.
\end{theorem}

\begin{proof}
    \begin{lemma}
        $t \mapsto (-1)^t$ is an injective map on $(-1, 1]$.
    \end{lemma}

    \begin{proof}
        Omitted.
    \end{proof}

    This means $(-1)^t = (-1)^{t_0}$ and $(-1)^s = (-1)^{s_0}$
    for some $t_0, s_0 \in (-1, 1]$.
    This means that if $t = s$, then by the lemma, $t_0 = s_0$,
    so $t \equiv s \pmod{2}$.
    
\end{proof}

\subsection{Further Properties of $\cis$}

Here are more theorems.

\begin{theorem}
    \[ \lim_{\theta \to 0} \frac{\sin \theta}{\theta} = 1. \]
\end{theorem}

\begin{theorem}
    \[ e^{i \theta} = \cis \theta. \]
\end{theorem}

Using section three we can see this corallary.
\begin{cor}
    If $\cis{\theta} = \cis{\phi}$, then $\theta \equiv \pi \pmod{2\pi}$.
\end{cor}
\emph{Hint}: Use the fact that $\cos$ restricted to $[0,\pi]$ is continuous.



