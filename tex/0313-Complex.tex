\section{March 13, 2019}

\subsection{Classifying Singularities}

\begin{definition}
    If $f$ is analytic in a punctured disk $0 < |z-a| < r$,
    then we call $a$ an \term{isolated singularity} of $f$.
\end{definition}

There are a bunch of different singularities:
\begin{enumerate}
    \item A \textit{removable singularity} is a point $a \in \C$
        such that 
        \[ \lim_{z \to a } (z-a)f(z) = 0. \]
    \item A \textit{pole}\footnote{A pole
        can be a removable singularity if it has order $-1$.} is a point $a \in \C$ such that
        \[\lim_{z \to a} |f(z)| = \infty. \]
    \item An \textit{essential singularity}.  Discussed below.
\end{enumerate}

Consider the following two conditions:
\begin{align*}
    C_0(\alpha, f)&: \;\;\;\; |z-a|^{\alpha} |f(z)| \to 0 \;\;\;\; 
    \text{as $z \to a$,} \\
    C_{\infty}(\alpha, f)&: \;\;\;\; |z-a|^{\alpha} |f(z)| \to \infty 
    \;\;\;\; \text{as $z \to a$.}
\end{align*}

Here are some consequences of the two conditions.
\begin{enumerate}
    \item If $C_0(\alpha, f)$ holds for some $\alpha$, then it
        $C_0(\beta, f)$ holds for all $\beta \ge \alpha$.
    \item If $C_{\infty}(\alpha, f)$ holds for some $\alpha$, then it
        $C_{\infty}(\beta, f)$ holds for all $\beta \le \alpha$.
    \item If $C_0(\alpha, f)$ holds for some $\alpha \in \C$,
        then $C_0(n, f)$ holds for some $n \in \Z$.
    \item If $C_{\infty}(\alpha, f)$ holds for some $\alpha \in \C$,
        then $C_{\infty}(n, f)$ holds for some $n \in \Z$.
\end{enumerate}
The first two are called \term{algebraic singularities}.

\begin{exercise}
    Prove each of these consequences.
\end{exercise}

\begin{theorem}
    If an analytic function $f$ is not identically $0$ and
    $C_0(\alpha, f)$ holds for some $\alpha \in \R$, then
    $h = \ord f$ acts as a \textit{separator}:
    \begin{center}
        If $\beta > h$, then $C_0(\beta, f)$ holds. \\
        If $\beta < h$, then $C_{\infty}(\beta, f)$ holds. \\
        If $\beta = h$, then neither $C_0$ nor $C_{\infty}$ hold.
    \end{center}
\end{theorem}

Rewritten version:
\begin{theorem}
    If $f$ is not identically zero and $f$ has an isolated
    singularity at $z=a$, then \textit{either}:
    \begin{enumerate}
        \item There is an integer $h$ such that all $\alpha > h$
            satisfy $C_{0}(\alpha)$ while all $\alpha < h$
            satisfy $C_{\infty}(\alpha)$, and
            \[ f(z) = {(z-a)}^{-h} \phi(z) \]
            in $D_r(a) \setminus \{a\}$ and $\phi(a) \notin
            \{0, \infty \}$.
        \item Neither $C_0(\alpha)$ nor $C_{\infty}(\alpha)$
            holds for any $\alpha \in \R$.
    \end{enumerate}
\end{theorem}
All singularities fall into either of these categories.
If a singularity is in the first category, it is algebraic,
and if it is in the second category, it is essential.



\subsection{Essential Singularities}

\begin{definition}
    An \term{essential singularity} is a point $a$ where neither
    $C_0(\alpha, f)$ nor $C_{\infty} (\alpha, f)$ hold for
    all $\alpha \in \R$.
\end{definition}

\begin{theorem}[Casorati-Weirstrass]
    If $a$ is an essential singularity of $f$,
    for any $A \in \C$ and any $\delta > 0$ and $\varepsilon > 0$,
    there exist points $z \in D_{\delta}(a) \setminus \{a\}$
    such that 
    \[ |f(z) - A| < \varepsilon. \]
    In other words, within an any positive radius disk of an
    essential singularity, we can get arbitrarily close to
    any complex number.
\end{theorem}

\begin{proof}
    Assume for sake of contradiction there exists $A_0 \in \C$
    and $\delta_0, \varepsilon_0 > 0$ such that for all
    $z \in D_{\delta_0}(a) \setminus \{a\}$, 
    \[ |f(z) - A_0 | \ge \varepsilon_0. \]
    Let $g(z) = f(z) - A_0$.
    Since $g$ is bounded away from $0$,
    \[ \lim_{z \to a} |z-a|^{-1} |g(z)| = 0, \]
    that is, $C_{\infty}(-1)$ holds for $g$.
    Therefore, there exists a separating integer $h$ for the
    function $g$ and thus a $\beta > 0$ such that $C_0(\beta, g)$
    holds:
    \[ |z-a|^{\beta}|g(z)| \to 0 \;\;\;\; \text{as $z \to a$.} \]
    However, this means
    \begin{align*}
        0 \ge |z-a|^{\beta} |f(z)| = |z-a|^{\beta}|g(z) + A_0|
        \le |z-a|^{\beta}|g(z)| + |z-a|^{\beta}|A_0| \to
        0 + 0 = 0.
    \end{align*}
    Therefore, $C_0(\beta, f)$ holds, which is a contradiction.
\end{proof}

\begin{theorem}
    If an entire function $f$ has a 
    nonessential (algebraic) singularity at $\infty$,
    then $f$ has to be polynomial.
\end{theorem}
What we mean by $f$ having a nonessential singularity at
$z = \infty$ is that $f(1/z)$
has a nonessential singularity at $z = 0$.

\begin{proof}
    Let $g(z) = f(1/z)$.
    A claim: we can write $g$ as a sum of an entire function
    $\psi$ and a singular part $S_0g$.
    This claim is left as an exercise.
    Then
    \[ g(z) = (S_0g)(z) + \psi(z). \]
    Therefore,
    \[ f(z) = (S_0g)\left(\frac{1}{z}\right) + \psi\left(\frac{1}{z}\right). \]
    Notice that $S_0g$ is a polynomial in $z$, let it be $P(z)$.
    Also let $\theta(z) = \psi(1/z)$.
    We claim $\theta$ is entire.
    This is equivalent to showing $\theta$ has a removable
    singularity at $z=0$.
    Send $z \to \infty$, so $1/z \to 0$.
    Note that $g(z) = f(1/z) \to f(0)$ and 
    $(S_0g)(z) = P(1/z) \to P(0)$ by continuity.
    Therefore $\theta$ is entire.
    By Liouville's, $\theta$ is constant.
    This means $f$ is polynomial, namely, $f = P$.
    %Note that $g$ has finite order at $z = 0$, so we can write
    %\[ g(z) = z^{-h} \phi(z) \]
    %If we write the Taylor Series of $\phi$ we find
    %\[ g(z) = c_{-h} z^{-h} + \cdots + c_1 z^{-1} + c_0 + 
    %c_1 z + \cdots.\]
    %Let $\psi(z) = c_0 + c_1z + \cdots$.
    %We claim $\psi$ is bounded.
    %Choose $\varepsilon > 0$.
    %For $|z| < \varepsilon$, $\psi$ is bounded because we
    %can choose it to be arbitrarily close to $c_0$.
    %For $|z| > \varepsilon$, $g(z) = f(1/z)$ is bounded
    %because $g$ achieves a maximum on the disk $\frac{1}{|z|} < 
    %\varepsilon$.  
    %Moreover, $|c_{-h}z^{-h} + \cdots + c_{-1}z^{-1}|$ is 
    %bounded by similar reasoning.
    %Since the difference of two bounded quantities is bounded,
    %$\psi$ is bounded for all $z \in \C$.
    %Therefore, $\psi$ is entire and bounded, so it is 
    %constant by Liouville's Theorem.
    %Therefore, $g$ is polynomial in $1/z$ so $f$ is polynomial
    %in $z$.
\end{proof}

\begin{theorem}
    If $f$ is entire and $|f(z)| \le A|z|^{\alpha}$ for
    $|z| \ge R$, then $f$ is polynomial.
\end{theorem}

\begin{proof}
    Let $n = \ceil{\alpha}$.
    We show the $n$th derivative of $f$ is identically $0$.
    Select an $a \in \C$ such that $|a| \le 1$.
    By Cauchy's Integral Formula for Derivatives:
    \begin{align*}
        |f^{n+1}(a)| &\le \left| \frac{(n+1)!}{2\pi i} \oint_{C_R(0)}
        \frac{f(w)}{{(w-a)}^{n+2}} \, dw \right| \\
        &\le (n+1)! \cdot R \cdot \max_{w \in C_R(0)} 
        \left|\frac{f(w)}{{(w-a)}^{n+2}}\right| \tag{ML Theorem}
    \end{align*}
    Now note $|f(w)| \le A|w|^{\alpha} \le Aw^n \le AR^n$ and
    $|a| \le 1$.
    This means as $R \to \infty$:
    \[ |f^{n+1}(a) \le (n+1)! \cdot R \cdot \frac{AR^n}{{(R-1)}^{n+1}} \to 0. \]
    Since the $f^{n+1}$ is identically $0$ on a disk,
    $f^{n+1}$ is identically $0$.
    Therefore, $f$ is polynomial.
\end{proof}


\begin{theorem}
    If $f$ is analytic in region $U$ and if $f$ vanishes on a
    subset $S \subseteq U$ such that $S$ has an accumulation
    point $p \in U$ ($p$ need not be in $S$), then $f \equiv 0$.
\end{theorem}

\begin{proof}
    If $f \ne 0$, then the roots of $f$ are isolated.
    By continuity $p$ is a root of $f$ because it is approached
    by roots in $S$.
    This is a contradiction beause we know roots of analytic
    functions are isolated.
\end{proof}

\begin{cor}
    If $f, g$ are analytic on $U$ and $f=g$ on 
    a \textit{subregion}
    $S \subseteq U$
    with an accumulation point $p \in U$, then $f \equiv g$ 
    on $U$.
\end{cor}


