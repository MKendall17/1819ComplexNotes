\section{Sept. 18, 2018}

\subsection{Basic Complex 
Number Properties}

\begin{definition}
    A \term{field} is a set $F$ together with operations $+,*$ such that
    \begin{itemize}
        \item $(F,+)$ is an \term{abelian}\footnote{commutative} group
        \item $(F \setminus \{ 0 \}, *)$ is an abelian group
    \end{itemize}
\end{definition}

\begin{example}
    $\R$ is a field.
\end{example}

\noindent
Let
\[ \C = \begin{cases}
(a,b) & b \ne 0, \\
a & b = 0
\end{cases}
\]

\noindent
Write $(a,b) = a(1,0) + b(0,1) = a\mathbf{1} + bi \in \R^2$
In this notation we define addition and multiplication:
\begin{align*}
    (a+bi) + (c+di) &:= (a+c) + (b+d)i \\
    (a+bi)(c+di) &:= (ac - bd) + (ad + bc)i \\
\end{align*}

\begin{definition}
    The \term{conjugate} of $z = a + bi$ is $\overline{z} := a - bi$.
\end{definition}

\begin{definition}
    The \term{norm} of $z$ is defined to be $|z| := \sqrt{a^2 + b^2}$.
\end{definition}

\begin{definition}
    The \term{reciprocal} of $z = a+bi$ is defined to be
    \[ \frac{1}{z} := \frac{a}{a^2 + b^2} + i \frac{-b}{a^2 + b^2}. \]
\end{definition}

\begin{definition}
    $z/w := z \cdot 1/w$.
\end{definition}

\begin{theorem}
    $z\overline{z} = |z|^2$.
\end{theorem}
\begin{proof}
    Let $z = a+bi$.  Then
    \[ z\overline{z} = (a+bi)(\overline{a+bi}) = (a+bi)(a-bi)
    = a^2 + b^2 = |z|^2. \]
\end{proof}

\begin{theorem}[Triangle Inequality]
$    |z + w| \le |z| + |w|$
\end{theorem}

\begin{cor}
    $|\real{z}| \le |z|$ and $|\imag{z}| \le |z|$.
\end{cor}

\begin{definition}
    The \term{distance} between complex numbers $z$ and $w$ is defined as
    \[ {\text{dist}}(z,w) := |z - w| \]
\end{definition}

\noindent
There are many important theorems involving conjugation and norms:
\begin{itemize}
    \item $\overline{z+w} = \overline{z} + \overline{w}$

    \item $\overline{zw} = \overline{z} \cdot \overline{w}$

    \item $\overline{\overline{z}} = z$

    \item $|zw| = |z| \cdot |w|$.

    \item $z = \overline{z}$ iff $z \in \R$ and $z = -\overline{z}$ iff $z \in i\R$
\end{itemize}




%\end{document}

