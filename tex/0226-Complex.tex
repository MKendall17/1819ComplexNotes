\section{Feb. 26, 2019}

\subsection{Order of Zeros}

\begin{theorem}
    Assume $f$ is analytic in region $U$ and fix $a \in U$ such that
    \[ 0 = f(a) = f'(a) = \cdots = f^{(n)}(a) = \cdots \]
    Then $f(z) \equiv 0$ in some open disk $D_r(a)$ with
    $r > 0$.
\end{theorem}

\begin{proof}
    By our discussion of Taylor Series, there exists $r > 0$
    such that for all $z \in D_r(a)$,
    \[ f(z) = \sum_{k=0}^{\infty} \frac{f^{(k)}(a)}{k!}(z-a)^k
    = 0. \]
    Since each of the derivatives at $a$ are $0$, $f \equiv 0$ 
    in $D_r(a)$.
\end{proof}

We can do better though.

\begin{theorem}
    If there exists a point where all of $f$'s derivatives 
    vanish, then $f$ is identically $0$ on all of $U$.
\end{theorem}

\begin{proof}
    Let
    \begin{align*}
        E_1 &= \{ z \in U \, | \, 0 = f(z) = f'(z) = \cdots \}, \\
        E_2 &= \{ z \in U \, | \, f(z) \ne 0 \text{\ or\ }
        f'(z) \ne 0 \text{\ or\ } \cdots \}.
    \end{align*}
    Note that $E_1$ and $E_2$ are open and $E_1 \cap E_2 = \emptyset$.

    \noindent
    Let's now prove that $E_1$ and $E_2$ are both open.
    $E_1$ is closed because of our previous theorem.
    $E_2$ is closed because $f$ is continuous, so we can find
    a disk around $z$ such that $f^{(k)}(z) \ne 0$.

    \noindent
    Since $U$ is connected and $E_1$ and $E_2$ are open and disjoint,
    we claim that either $E_1 = \emptyset$ and $E_2 = \emptyset$.

    \begin{lemma}
        Suppose $E_1$ and $E_2$ are open sets such that they are
        open, disjoint, and their union is an open connected
        set $U$.
        
        \noindent
        Then either $E_1 = \emptyset$ and $E_2 = \emptyset$.
    \end{lemma}

    \begin{proof}
        Suppose for sake of contradiction that $E_1$
        and $E_2$ are nonempty.
        Choose $p \in E_1$ and $q \in E_2$ such that
        There exists a path
        $\gamma : [0,1] \to U$ with $\gamma(0) = p$
        and $\gamma(1) = q$.
        Let
        \[ T = \left\{ t \in [0,1] \, | \, \ran{\gamma\big|_{[0, t]}} 
            \subseteq E_1 \right\}.
        \]

        Since $0 \in T$, $T \ne \emptyset$.  
        This means $T$ has a supremum; let it be $t_0$.
        Since $\gamma$ is continuous, define $z_0 = \gamma(t_0) \in U$.
        We split the next part into two cases each with two
        subcases.
        \begin{enumerate}
            \item $z_0 \in E_1$.  There are two cases to consider.
                \begin{enumerate}
                    \item Suppose $t_0 = 1$.
                        Note that $\gamma(t_0) = \gamma(1) = q$.
                        This means $q \in E_1$, a contradiction.

                    \item Suppose $t_0 < 1$.
                        This means we can extend the supremum 
                        because $t_0$
                        is less than $1$.
                \end{enumerate}

            \item $z_0 \in E_2$
                This is left as an exercise to the reader.
        \end{enumerate}

    \end{proof}

        The lemma completes the theorem, so one of $E_1$
        or $E_2$ is empty.


\end{proof}

\begin{exercise}
    Prove case $2$ of the lemma in the previous theorem.
\end{exercise}

From this comes the following definition.

\begin{definition}
    If $f$ is analytic but not identically $0$ in $U$, 
    then for each $a \in U$, there is a least nonnegative
    integer $h$ such that $f^{(h)}(a) \ne 0$.
    This minimal $h \ge 0$ is called the \term{algebraic
    order} of $f$ at $a$, denoted
    \[ h = \ord{a}. \]
\end{definition}

So if $f$ is not identically $0$ on $U$, then every
root of $U$ of $f$ has finite order.

\begin{theorem}
    Say $a$ is a root of order $h \ge 1$.
    Then
    \[ f(z) = (z-a)^h f_h(z), \]
    where $f_h(z)$ is also analytic in $U$ and
    $f_h(a) \ne 0$.
\end{theorem}

\begin{proof}
    Write out the Taylor Series of $f$ centered $a$:
    \[ f(z) = f(a) + f'(a)(z-a) + \cdots + \frac{f^{(h-1)}(a)}{(h-1)!} (z-a)^{h-1} + (z-a)^h f_h(z)
        = (z-a)^h f_h(z),
    \]
    for some analytic function $f_h$.
    Also,
    \[ f_h(a) = \frac{f^{(h)}(a)}{h!} \ne 0. \]
\end{proof}

There is a useful corollary to this.

\begin{cor}
    Note that this means the zeros of $f$ are \textit{isolated}:
    around every $z_0$ such that $f(z_0) = 0$, there is a disk
    of positive radius $r$ such that $f(z) \ne 0$ on
    $D_r(z_0) \setminus \{ z_0 \}$.
\end{cor}


\subsection{Meromorphic functions}

\begin{definition}
    A function $f$ is called \term{meromorphic} in a region
    $U$ if it is analytic in $U$ except at isolated points
    which are \term{poles}:
    \[ \lim_{z \to a_j} |f(z)| = \infty 
        \tag{for $a_j$ running over a countable index set $I$}
    \]

    We also need $a_j$ to be an \term{isolated singularity}:
    there exists $r_j>0$ such that $D_{r_j}(a_j)$
    contains no other singularities/poles.
    So on $0 < |z-a_j| < r_j$, $f$ is analytic.

\end{definition}

\begin{exercise}
    The ratio of $f$ to $g$ is meromorphic in $U$ provied
    $g$ is not identically $0$.
%    $g \nequiv 0$.
\end{exercise}

Just like the number of zeros of an analytic function is countable,
so are the poles on a meromorphic function.

\begin{theorem}
    The number of poles of a meromorphic function
    is at most countably infinite.
\end{theorem}

\begin{proof}
    Make the disks of each pole pairwise disjoint.
    We can do this because the poles are isolated singularities.
    Pick a rational point in each.
    Therefore, there are a countable number of disks,
    so there are a countable number of poles.
\end{proof}

Now let's try to extend algebraic order to meromorphic 
functions.

\begin{theorem}
    Say $f$ is meromorphic on region $U$.
    Then there exists a positive integer $h$ such that
    \[ f(z) = (z-a)^{-h} g(z), \]
    where $g$ is meromorphic, analytic, and nonzero at $a$.
\end{theorem}

\begin{proof}
    Take $r > 0$ small enough such that $f$ is analytic
    on $0 < |z-a| < r$.
    Reduce $r$ if necessary such that $f(z) \ne 0$ for
    $0 < |z-a| < r$.
    Let 
    \[ l(z) = \frac{1}{f(z)}, \]
    defined on the punctured disk.  
    We know $l$ is analytic on the punctured disk because
    $f$ is analytic and nonzero.
    Also,
    \[ \left| \lim_{z \to a} l(z) \right|
        = \lim_{z \to a} |l(z)|
        = \lim_{z \to a} \left| \frac{1}{f(z)}\right|
        = 0.
    \]
    Therefore, $l$ approaches $0$ as $z \to a$.
    This means $a$ is a removable singularity of $l$.
    Extend $l$ to analytic function such that $l(a) = 0$.
    Since l is not identicall $0$, it has finite order, so we 
    % $l \nequiv 0$, $l$ has finite order $h$, so we
    let $\sigma$ to be an analytic function such that
    \[ l(z) = (z-a)^h \sigma(z), \]
    where $\sigma(z) \ne 0$.
    If we also let $g(z) = \frac{1}{l(z)}$, then
    \[ f(z) = (z-a)^{-h} g(z), \]
    Note $g(a) = \frac{1}{\sigma(a)} \notin \{0, \infty \}$.
    We can also extend $g$ so that $g(z) = (z-a)^hf(z)$ 
    on all of $U$.
    Now $g$ is analytic in $D_r(a)$ and meromorphic on $U$
    (with all of $f$'s poles except $a$).
\end{proof}

\begin{definition}
    The \term{order} of a meromorphic function $f$ at 
    $a$ is 
    \[ \ord(f, a) = -h \]
    for some integer $h$.
\end{definition}

Some properties:
\begin{enumerate}
    \item $\ord(f, a) > 0$ when $a$ is a root.
    \item $\ord(f, a) < 0$ when $a$ is a pole.
    \item $\ord(f, a) = 0$ when $f(a)$ is neither $0$ nor
        $\infty$.
\end{enumerate}



\subsection{Nature of Zeros and Poles}

Call $r$ a \emph{simple root} of $f$ to be a root of 
multiplicity $1$: $f(r) = 0$ and $f'(r) \ne 0$.

\begin{definition}
    We say $z_0 \in \hat{\C}$ is an \term{accumulation point}
    of a subsequence of poles $(a_n)_{n=1}^{\infty}$ if
    \[ z_0 = \lim_{k \to \infty} a_{n_k} \;\;\;\; \text{for 
    some subsequence $n_1 < n_2 < \cdots$}. \]
\end{definition}

\begin{theorem}
    If $f$ is analytic and it has countably many roots 
    $a_1, a_2, \ldots$ which accumulate to $z_0$, then 
    $z_0 \in \bd U$.
\end{theorem}

\begin{proof}
    Certainly $z_0$ cannot be in the exterior of $U$ (because then
    you can draw a disk around $z_0$ contained in the exterior
    and you need points in $U$ to approach it).
    If $z_0 \in U$, then $f(a_{n_k}) \to f(z_0)$ since $f$ is
    continuous and $f(a_{n_k}) = 0$ for all $k$ so $f(a_{n_k})
    \to 0$.

    This means $f(z_0) = 0$, 
    so $z_0$ is a root that is not isolated.
    That's a contradiction, so $z_0 \in \bd U$.
\end{proof}


\begin{theorem}
    If $f$ is meromorphic on $\C$ and has infinitely many
    poles, then it has an accumulation point at $\infty$.
\end{theorem}

\begin{proof}
    If $(a_n)_{n=1}^{\infty}$ is bounded, then by
    Bolzano-Weirstrass there exists a convergent subsequence
    $(a_{n_k})_{k=1}^{\infty}$ and its limit would be $z_0$.
    This means $z_0 \in \C$, but this can't be because
    $z_0 \notin \bd \C$.
\end{proof}







