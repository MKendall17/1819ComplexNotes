\section{June 4, 2019}
\begin{definition}
    A cycle $\beta$ in $U$ \term{bounds} a subregion $V \subseteq U$,
    write $\beta = \partial V$ if
    \[ n(\beta, z) = 
        \begin{cases} 1 & \text{if $z \in V$}, \\
            0 & \text{if $z \notin V \cup \ran{\beta}$}.
        \end{cases}
        \]
\end{definition}

\subsection{Argument Principle for Meromorphic Functions}
\begin{theorem}
    Let $f$ be nonconstant and meromorphic in region $U$.
    Let $f$ have roots $a_1, \ldots$ and poles $b_1, \ldots$ in $U$.
    Let $\beta$ be a cycle in $U$ such that $\beta \sim 0 \pmod{U}$,
    and $a_i, b_j \notin \ran \beta$.
    Then
    \[ \frac{1}{2\pi i} \oint_{\beta} \frac{f'}{f} = \sum_n n(\beta, a_n)
    \ord (f, a_n) + \sum_{m} n(\beta, b_m) \ord(f, b_m). \]
\end{theorem}
A couple of notes:
\begin{enumerate}
    \item $\ord(f, a_n) > 0$ and $\ord (f, b_m) < 0$.
    \item Both sums are finite.  This is because we can consider
        $V = \{z \, | \, n(\beta, z) = 0\}$, which is open and contains
        the exterior of some disk.
        This means $\{z \, | \, n(\beta, z) \ne 0\}$ is closed and bounded.
        If there are an infinite number of roots, for example, then
        by Bolzano Weirstrass there exists a subsequence of those roots
        that converges to a point $p \in \bd U$.
        Since $\beta$ is closed, $n(\beta, a_i) = 0$ for all
        but a finite number of roots.
        Therefore there are a finite number of nonzero terms.
\end{enumerate}

\begin{proof}
    Apply the Residue Theorem to $g = f'/f$.
\end{proof}

\subsection{Rouche's Theorem and the Dogwalker Theorem}
\begin{theorem}[Rouche's Theorem]
    Let $f, g$ be analytic in region $U$.
    Let $\beta$ be a cycle in $U$ such that $\beta \sim 0 \pmod{U}$.
    For $z \in \ran \beta$, suppose $|f(z)-g(z)| < |g(z)|$.
    Also let $\beta = \partial V$, where $V$ is a region.
    Then
    \[ Z(f, V) = Z(g, V), \]
    where $Z(h, V)$ is the number of roots of $h$ in $V$.
\end{theorem}

\begin{proof}
    Here are the steps.
    \begin{enumerate}
        \item On $\ran \beta$, $f(z) \ne 0$ and $g(z) \ne 0$.
        \item On $\ran \beta$, $\left|\frac{f(z)}{g(z)}-1\right| < 1$.
        \item $F(z) := \frac{f(z)}{g(z)}$ is meromorphic in $U$.
            By the Argument Principle,
            \[ \frac{1}{2\pi i} \oint_{\beta} \frac{F'}{F} =
                \frac{1}{2\pi i} \oint_{\beta} \frac{f'}{f} - \frac{g'}{g}
            = Z(f, V) - Z(g, V). \]
            Also note
            \[ \oint_{\beta} \frac{F'(z)}{F(z)} \; dz = 
                \oint_{F \circ \beta} \frac{dw}{w} = n(F \circ \beta, 0).
            \]
            But by $2$, $\ran (F \circ \beta) \subseteq D_1(1)$, and
            since $0 \notin D_1(1)$, $n(F \circ \beta, 0) = 0$.
            This implies $Z(f, V) - Z(g, V) = 0$, so $Z(f, V) = Z(g, V)$.
    \end{enumerate}
\end{proof}

\begin{theorem}
    Let $\gamma(t)$ and $\delta(t)$ be loops defined on $t \in [0,1]$
    such that $|\gamma(t) - \delta(t)|
    < |\gamma(t)|$ for all $t \in [0,1]$.
    Then
    \[ n(\gamma, 0) = n(\delta, 0). \]
\end{theorem}

\begin{exercise*}
    Figure out the proof of the Dogwalker Theorem.
\end{exercise*}

\subsection{Homotopy of Loops}

\begin{definition}
    Given two loops $\gamma_0$ and $\gamma_1$ parameterized
    on $[0,1]$, a $\term{homotopy of loops}$ joining $\gamma_0$ to 
    $\gamma_1$ is a continuous map $H : [0,1] \times [0,1] 
    \subseteq \R^2 \to \C$ such that
    \begin{itemize}
        \item $H(t, 0) \equiv \gamma_0(t)$ for all $t \in [0,1]$.
        \item $H(t, 1) \equiv \gamma_1(t)$ for all $t \in [0,1]$.
        \item $H(0, s) \equiv H(1, s)$ for all $s \in [0,1]$.
    \end{itemize}
\end{definition}

\begin{definition}
    A \term{smooth homotopy of loops} joining $\gamma_0$ to $\gamma_1$
    requires $H$ to be jointly $C^1$, or $\frac{\partial H}{\partial x}$ and $\frac{\partial H}{\partial y}$ are continuous.
\end{definition}

\begin{theorem}
    Suppose $\gamma_0$ and $\gamma_1$ are loops avoiding 
    $a \in \C$ and there exists a smooth homotopy $H$ of loops
    joining $\gamma_0$ to $\gamma_1$, where $a \notin \ran H$.
    Then
    \[ n(\gamma_0, a) = n(\gamma_1, a). \]
\end{theorem}

\begin{proof}
    We use the following theorem.
    \begin{theorem*}[Uniform Continuity Theorem]
        A continuous function on a compact set is \textit{uniformly}
        continuous.
    \end{theorem*}
    Note that
    \[ n(\gamma_s, a) = \frac{1}{2\pi i} \oint_{\gamma_s} 
\frac{dz}{z-a} = \frac{1}{2\pi i} \int_0^1 \frac{\frac{\partial H}{\partial t}(t, s)}{H(t, s) - a} \; dt. \]
Note that the integrand is uniformly continous on ${[0,1]}^2$ because it
    is a function on a compact set and both the numerator and
    denominator are continuous by assumption.
    Therefore we can use the ML theorem to bound the modulus
    of the integral to show $g(s)=n(\gamma_s, a)$ is continuous.
    
    On the other hand, $g$ only takes on integer values,
    so $g$ must be constant.
    Therefore, $g(0) = g(1)$, as desired.
\end{proof}

\subsection{Mittag Lefler's Theorem}
\begin{theorem}
    Let $f$ be meromorphic in $\C$ with poles $b_1, b_2, \ldots,
    b_{\nu}, \ldots$ with $b_{\nu} \ne 0$.
    Then there exists polynomials $p_{\nu}$ such that if the
    singular part of $f$ at $b_{\nu}$ is $P(1/(z-b_{\nu}))$
    and there exists an entire function $g$ such that
\[ f(z) = \sum_{\nu \ge 1} \left[ P_{\nu} \left( \frac{1}{z-b_{\nu}}\right) - p_{\nu}(z)\right] + g(z), \]
    for all $z \ne b_{\nu}$.
\end{theorem}
