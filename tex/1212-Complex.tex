\section{Dec. 12, 2018}

\subsection{Independence of Path}

We say $f(z)$ is \term{independent of path} if
\[ \int_{\gamma} f(z) \; dz = G(\gamma(0), \gamma(N)) \]
for some function $G$, where $[0,N] = \dom{\gamma}$ for 
\textit{all} paths $\gamma$ in region $U$.

\begin{theorem}[Independence of Path - IOP]
    Suppose $f(z)$ is continuous and has an antiderivative $F(z)$ throughout the region $U$.
    Let $F(z)$ be the antiderivative of $f$.
    Then
    \[ \int_{\gamma} f(z) \; dz = F(\gamma(N)) - F(\gamma(0)). \]
\end{theorem}

\begin{proof}
    The proof is another string of integrals.
    \begin{align*}
        \int_{\gamma} f(z) \; dz &= \int_0^N f(\gamma(t)) \gamma'(t) \; dt \\
        &= \int_0^N F'(\gamma(t)) \gamma'(t) \; dt \\
        &= \int_0^N \frac{d}{dt} (F(\gamma(t))) \; dt \\
        &= F(\gamma(N)) - F(\gamma(0)).
    \end{align*}
    This completes the proof.
\end{proof}

Here is the Fundamental Theorem of Complex Calculus.
\begin{theorem}[Fundamental Theorem of Complex Calculus - FTCC]
    Given a function $F$ such that $F$ is continuous on an open
    set $U$ containing $\ran{\gamma}$, then
    \[ \int_{\gamma} F'(z) \; dz = F(\gamma_{\text{term}}) -
    F(\gamma_{\text{init}}). \]
\end{theorem}

Now for the converse of IOP.
\begin{theorem}[Converse of IOP]
    We prove if $\int_{\gamma} f(z) \; dz$ IOP, then there exists
    an analytic function $F$ on $U$ such that $F'(z) = f(z)$ for
    all $z \in U$.
\end{theorem}

\begin{proof}
    Fix a point $p \in \C$ and for an arbitrary point $z \in \C$,
    let $\gamma_{p, z}$ be a path from $p$ to $z$.
    Define
    \[ F(z) = \int_{\gamma_{p, z}} f(z) \; dz. \]
    Note that $F$ is well defined because $f$ is independent
    of path.
    We show $F$ is analytic by using
    the converse of the Cauchy-Riemann equations.

    \noindent
    The first step is to show that
    \[ \frac{\partial F}{\partial x}(z) = f(z). \]
    For some real number $r$,
    let $\eta_r$ be the path from $z$ to $z + r$.
    Now let's try to approximate the difference quotient
    to $f(z)$:
    \begin{align*}
        \left| \frac{F(z+r) - F(z)}{r} - f(z)\right| &= 
        \left| \frac{1}{r}\left( \int_{\gamma_{p, z+r}} f(w) \; dw - 
        \int_{\gamma_{p, z}} f(w) \; dw \right)  - f(z) \right| \\
        &= \left| \frac{1}{r}\left( \int_{[\gamma_{p, z}] + [\eta_r]} f(w) \; dw - 
        \int_{\gamma_{p, z}} f(w) \; dw \right)  - f(z) \right| \tag{Independence of Path} \\
        &= \left| \frac{1}{r} \int_{\eta_r} f(w) \; dw - 
        \frac{1}{r} \int_{\eta_r} f(z) \; dw  \right| \tag{Concatenation of Paths} \\
        &\le \frac{1}{|r|} |r| \max_{z \in \eta_r} |f(w) - f(z)|
        \to 0, \tag{ML and Continuity of $f$}
    \end{align*}
    as $r \to 0$.
    Therefore, $\frac{\partial F}{\partial x}(z) = f(z)$.

    \noindent
    It is an execrise to show
    \[ \frac{\partial f}{\partial y} (z) = if(z). \]
    This means the Cauchy-Riemann equation is satisfied:
    \[ \frac{\partial f}{\partial y} = i \frac{\partial f}{\partial x}. \]
    Finally, we know the partials are continuous because
    $f(z)$ and $if(z)$ are continuous.
    Therefore, $F$ is analytic, so there does exist an 
    antiderivative of $f$.

\end{proof}

\begin{exercise}
    Finish the proof of the converse of IOP by showing
    \[ \frac{\partial f}{\partial y} (z) = if(z). \]
\end{exercise}




