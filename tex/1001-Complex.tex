\section{Oct. 1, 2018}

\subsection{Definition of Principal Argument}

\begin{definition}
    The \term{principal argument} of a complex number 
    $z = x + iy$ is defined to be
    \[ {\Arg}(z) := \begin{cases}
    \pi & \text{if $z \in \R^-$} \\
    {\sgn}(y) \arccos{\frac{x}{\sqrt{x^2+y^2}}} & \text{otherwise}
    \end{cases} \]
\end{definition}

Here's a basic consequence
of our definition.

\begin{theorem}
    $-\pi < {\Arg}(z) \le \pi$.
\end{theorem}

\subsection{Properties of Principal Argument}

\begin{theorem}[Polar Form Theorem]
    \[ z = |z| \Arg(z). \] 
\end{theorem}

\begin{theorem}
    \[ \Arg(zw) \equiv \Arg(z) + \Arg(w) \pmod{2\pi}. \]
\end{theorem}



