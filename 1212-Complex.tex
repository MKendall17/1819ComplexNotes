\documentclass[notes]{subfile}

\begin{document}

\section{Dec. 12, 2018}

\subsection{Independence of Path}

We say $f(z)$ is \term{independent of path} if
\[ \int_{\gamma} f(z) \; dz = G(\gamma(0), \gamma(N)) \]
for some function $G$, where $[0,N] = \dom{\gamma}$ for 
\textit{all} paths $\gamma$ in region $U$.

\begin{theorem}[Independence of Path - IOP]
    Suppose $f(z)$ is continuous and has an antiderivative $F(z)$ throughout the region $U$.
    Let $F(z)$ be the antiderivative of $f$.
    Then
    \[ \int_{\gamma} f(z) \; dz = F(\gamma(N)) - F(\gamma(0)). \]
\end{theorem}

\begin{proof}
    The proof is another string of integrals.
    \begin{align*}
        \int_{\gamma} f(z) \; dz &= \int_0^N f(\gamma(t)) \gamma'(t) \; dt \\
        &= \int_0^N F'(\gamma(t)) \gamma'(t) \; dt \\
        &= \int_0^N \frac{d}{dt} (F(\gamma(t))) \; dt \\
        &= F(\gamma(N)) - F(\gamma(0)).
    \end{align*}
    This completes the proof.
\end{proof}

Here is the Fundamental Theorem of Complex Calculus.
\begin{theorem}[Fundamental Theorem of Complex Calculus - FTCC]
    Given a function $F$ such that $F$ is continuous on an open
    set $U$ containing $\ran{\gamma}$, then
    \[ \int_{\gamma} F'(z) \; dz = F(\gamma_{\text{term}}) -
    F(\gamma_{\text{init}}). \]
\end{theorem}

Now for the converse of IOP.
\begin{theorem}[Converse of IOP]
    We prove if $\int_{\gamma} f(z) \; dz$ IOP, then there exists
    an analytic function $F$ on $U$ such that $F'(z) = f(z)$ for
    all $z \in U$.
\end{theorem}

\begin{proof}
    Fix a point $p \in \C$ and for an arbitrary point $z \in \C$
    let $\gamma_{p, z}$ be a path from $p$ to $z$.
    Define
    \[ F(z) = \int_{\gamma_{p, z}} f(z) \; dz. \]
    Note that $F$ is well defined because $f$ is independent
    of path.
    We show $F$ is analytic by using
    the converse of the Cauchy-Riemann equations.

    \noindent
    The first step is to show that
    \[ \frac{\partial F}{\partial x}(z) = f(z). \]
    For some real number $r$,
    let $\eta_r$ be the path from $z$ to $z + r$.
    Now let's try to approximate the difference quotient
    to $f(z)$:
    \begin{align*}
        \left| \frac{F(z+r) - F(z)}{r} - f(z)\right| &= 
        \left| \frac{1}{r}\left( \int_{\gamma_{p, z+r}} f(w) \; dw - 
        \int_{\gamma_{p, z}} f(w) \; dw \right)  - f(z) \right| \\
        &= \left| \frac{1}{r}\left( \int_{[\gamma_{p, z}] + [\eta_r]} f(w) \; dw - 
        \int_{\gamma_{p, z}} f(w) \; dw \right)  - f(z) \right| \tag{Independence of Path} \\
        &= \left| \frac{1}{r} \int_{\eta_r} f(w) \; dw - 
        \frac{1}{r} \int_{\eta_r} f(z) \; dw  \right| \tag{Concatenation of Paths} \\
        &\le \frac{1}{|r|} |r| \max_{z \in \eta_r} |f(w) - f(z)|
        \to 0 \tag{ML and Continuity of $f$}
    \end{align*}
    Therefore, $\frac{\partial F}{\partial x}(z) = f(z)$.

    \noindent
    It is an execrise to show
    \[ \frac{\partial f}{\partial y} (z) = if(z). \]
    This means the Cauchy-Riemann equation is satisfied:
    \[ \frac{\partial f}{\partial y} = i \frac{\partial f}{\partial x}. \]
    Finally, we know the partials are continuous because
    $f(z)$ and $if(z)$ are continuous.
    Therefore, $F$ is analytic, so there does exist an 
    antiderivative of $f$.

\end{proof}

\begin{exercise}
    Finish the proof of the converse of IOP by showing
    \[ \frac{\partial f}{\partial y} (z) = if(z). \]
\end{exercise}


\subsection{Cauchy-Goursat Theorem}
First some preliminaries.

\noindent
Let 
\[ I = \int_{\gamma} (z-a)^n \; dz. \]
If $n \in \Z$ and $\gamma$  is a piecewise smooth loop, then
\begin{enumerate}
    \item For $n \ge 0$, $I = 0$ for any $a \in \C$.
    \item For $n \le -2$, provided $a \notin \ran{\gamma}$,
        $I = 0$.
    \item If $n = -1$,
        \[ I = \begin{cases}
                0 & \text{if $\ran{\gamma} \subseteq \C \setminus \R^-$.} \\
                2\pi i & \text{if $\gamma$ is a counterclockwise circle 
                centered at $a$.}
            \end{cases}
        \]
       
\end{enumerate}

Let's prove these one at a time.

\begin{enumerate}
    \item \textbf{work here}
    \item \textbf{work here}
    \item \textbf{work here}
\end{enumerate}

Now for the big theorem(s).

\begin{theorem}[Cauchy-Goursat I]
    Let $f(z)$ be a function analytic on the interior of 
    $R = [a,b] \times [c,d]$.
    Let $\Gamma(R)$ be the boundary of the rectangle traversed 
    counterclockwise starting from the bottom let corner.
    Then
    \[ \oint_{\Gamma(R)} f(z) \; dz = 0. \]
\end{theorem}

\begin{proof}
    For some path $S$, let 
    \[ I(S) = \int_{\Gamma(S)} f(z) \; dz \]
    and let $\Gamma(R^1), \Gamma(R^2), \Gamma(R^3), \Gamma(R^4)$
    be the counterclockwise paths of four congruent rectangle that cover $R$.
    
    \noindent
    Note that the boundaries of the four rectangles are cancecled when integrating, so
    \[ I(R) = I(R^1) + I(R^2) + I(R^3) + I(R^4). \]
    This means the lengths of one of
    $I(R^1), I(R^2), I(R^3), I(R^4)$ is at least
    one fourth the length of $I(R)$.
    Let $R_1$ be the rectangle such that $I(R_1) \ge \frac{1}{4} I(R)$.
    We define $R_2, R_3, \ldots$ similarly, where
    \[ R \supseteq R_1 \supseteq R_2 \supseteq \cdots
    \supseteq R_n \supseteq \cdots .\]
    In general,
    \[ |I(R_n) | \ge 4^{-n} |I(R)|. \]
    Note that there exists a $p \in R$ such that the interior
    of $R_n$ contains $p$ for all $n$.
    This is because the left $x$ coordinates of the rectangles
    for a nondecreasing sequence with an upper bound, namely $d$,
    and similarly, the right $x$ coordinates and left/right
    $y$ coordinates converge to a point.
    Now let's bound

    \[ I(R_n) = \int_{\Gamma(R_n)} f(z) \; dz. \]

    Note that
    \begin{align*}
        I(R_n) &= \int_{\Gamma(R_n)} f(z) \; dz \\
        &= \int_{\Gamma(R_n)} f(z) - f(p) - (z-p)f'(p) \; dz.
    \end{align*}
    The second equation comes from the fact that integrating
    a constant or a linear around a loop does not add anything.

    Now because the $R_n \to p$, we can choose $n$ large 
    enough so that for $z \in R_n$:
    \[ |f(z) - f(p) - (z-p)f'(p)| < \epsilon |z-p| \]
    because $f$ is analytic at $p$.
    
    Let $P$ be the perimeter of $R$ and $d$ be
    the length of the diagonal of $R$.
    We finish with the ML theorem:
    \begin{align*}
        I(R_n) &\le P \cdot 2^{-n} \max_{z \in R_n} |f(z) - f(p) - (z-p)f'(p)| \\
        &< P \cdot 2^{-n} \cdot \epsilon|z-p| \\
        &\le P \cdot 2^{-n} \cdot \epsilon 2^{-n} \\
        &= P \cdot 4^{-n} \epsilon.
    \end{align*}
    Therefore,
    \[ 4^{-n} |I(R)| \le |I(R_n)| < P \cdot 4^{-n}\epsilon \to 0 \]
    as $n \to \infty$.
    This is exactly what we desire, because this means
    \[ \oint_{\Gamma(R)} f(z) \; dz = 0. \]
    
\end{proof}

\begin{theorem} [Cauchy-Goursat II]
    Let $f$ be analytic in region $U$ containing a rectangle
    $R = [a,b] \times [c,d]$ except at finitely many points
    $p_1, p_2, \ldots, p_n$ such that for $1 \le i \le n$:
    \[ \lim_{z \to p_j} (z-p_j) f(z) = 0.\]
    Then
    \[ \oint_{\Gamma(R)} f(z) \; dz = 0. \]
\end{theorem}

\begin{proof}
    First, let's cut up $R$ into smaller rectangles such
    that each rectangle has at most one point
    where $f$ is not defined.
    Let's call one of those rectangles $S$ and the point
    at which $f$ is not defined $p$.
    We are left to show
    \[ I(S) = \oint_{\Gamma(S)} f(z) \; dz = 0, \]
    because when we sum over all $I(S)$, we will get $I(R)$.

    \noindent
    Let $P$ the perimeter of $S$ and $d$ be the diagonal
    of $S$.
    Note that by the limit condition, by cutting 
    up $S$ into more smaller rectangles, 
    we can restrict ourselves to points $z$ $\delta$
    away from $p$ such that
    \[ |(z-p)f(z)| < \epsilon. \tag{1} \]

    We can now bound $|I(s)|$:
    \begin{align*}
        |I(S)| &= \oint_{\Gamma(S)} f(z) \; dz \\
        &\le P \cdot \max_{z \in \text{bd}(S)} |f(z)| 
        \tag{ML Theorem}\\
        &< P \cdot \frac{\epsilon}{|z-p|} 
    \end{align*}
    Therefore, $I(S) \to 0$ as we get closer to $p$.
    This means
    \[ \oint_{\Gamma(R)} f(z) \; dz = 0, \]
    as desired.
\end{proof}

\begin{theorem}[Cauchy-Goursat III]
    Let $f$ be analytic in an open disk $\Delta = D_r(p)$
    except at finitely many points $p_1, p_2, \ldots, p_n$.
    Let $\gamma$ be \textit{any} loop in $\Delta \setminus 
    \{ p_1, p_2, \ldots, p_n \}$ such that for all of these points,
    \[ \lim_{z \to p_j} (z-p_j)f(z) = 0.\]
    Then
    \[ \oint_{\gamma} f(z) \; dz = 0. \]
\end{theorem}

\begin{proof}
    The first step is to show that $f$ has an antiderivative $F$.
    Once $f$ has an antiderivative, we would be done
    because $f$ would be independent of path, which means
    the loop integral is zero.

    \noindent
    For a fixed point $p \in \C$, let $\gamma_{p, z}$ be a standard\footnote{We spent a couple days in class trying to
        find an algorithm for a rectangular path.
    }
    rectangular path from $p$ to $z$. 
    Define, just like before:
    \[ F(z) = \int_{\gamma_{p, z}} f(w) \; dw .\]
    We show $F$ is the antiderivative of $f$ by showing
    $F$ satisfies the Cauchy-Riemann equations and $F$ 
    has continuous partials.
    Let's examine $\frac{\partial f}{\partial x}$ by 
    looking at the difference quotient for some real $r$:
    \begin{align*}
        D = 
        \left| \frac{F(z+r) - F(z)}{r} - f(z)\right| =
        \left| \frac{1}{r}\left( \int_{\gamma_{p, z+r}} f(w) \; dw - 
        \int_{\gamma_{p, z}} f(w) \; dw \right)  - f(z) \right| \\
    \end{align*}
    Note that Cauchy-Goursat I and II essentially state that
    when $f$ is analytic in $\Delta$ except at finitely many
    not too bad points, then $f$ is IOP when integrated
    over rectangular paths.

    \noindent
    We can now continue as we did when we proved converse of IOP:
    \begin{align*}
        D &= \left| \frac{1}{r}\left( \int_{[\gamma_{p, z}] + [\eta_r]} f(w) \; dw - 
        \int_{\gamma_{p, z}} f(w) \; dw \right)  - f(z) \right| \tag{Independence of Path} \\
        &= \left| \frac{1}{r} \int_{\eta_r} f(w) \; dw - 
        \frac{1}{r} \int_{\eta_r} f(z) \; dw  \right| \tag{Concatenation of Paths} \\
        &\le \frac{1}{|r|} |r| \max_{z \in \eta_r} |f(w) - f(z)|
        \to 0 \tag{ML and Continuity of $f$}
    \end{align*}
    This means $\frac{\partial f}{\partial x} = f(z)$.
    Similarly, $\frac{\partial f}{\partial y} = if(z)$.
    Moreover, $F$ has continuous partials because $f$ is analytic.
    
    \noindent
    Therefore, $F$ is the antiderivative of $f$, so
    $f$ is independent of path.
    This proves that
    \[ \oint_{\gamma} f(z) \; dz = 0. \]
\end{proof}



\end{document}
