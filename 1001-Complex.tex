%        File: 1001-Complex.tex
%     Created: Tue Jan 22 10:00 PM 2019 E
% Last Change: Tue Jan 22 10:00 PM 2019 E
%
\documentclass[notes]{subfile}
\begin{document}
\section{Oct. 1, 2018}

\begin{definition}
    The \term{principal argument} of a complex number $z$ is defined to be
    \[ {\text Arg}(z) := \begin{cases}
    \pi & \text{if $z \in \R^-$} \\
    {\text sgn}(y) \arccos{\frac{x}{\sqrt{x^2+y^2}}} & \text{otherwise}
    \end{cases} \]
\end{definition}

\begin{theorem}
    $-\pi < {\text Arg}(z) \le \pi$.
\end{theorem}

\begin{theorem}
    If $(-1)^t = (-1)^s$ for real numbers $t,s$, then $t \equiv s \pmod{2}$.
\end{theorem}

\begin{proof}
    \begin{lemma}
        $t \mapsto (-1)^t$ is one-to-one on $(-1,1]$.
    \end{lemma}

    \begin{proof}
        Let $t,s \in (-1,1]$ be two points.
        Let $\mathbb{S} = \{ z \in \C \; | \; |z| = 1 \}$ be the unit circle.
        We partition $\mathbb{S}$ into an upper semicircle $\mathbb{S}^+$, a lower semicircle $\mathbb{S}^-$ and the two points $\{1,-1\}$.

        If $t,s$ map to different semicircles, then we are okay.
        Otherwise, suppose $0<t<s<1$.  Then $\real{(-1)^t} > \real{(-1)^s}$.
        Now suppose $-1<t<s<0$.  Then $\real{(-1)^t} > \real{(-1)^s}$

        Therefore, $t \mapsto (-1)^t$ is one-to-one on $(-1,1]$.
    \end{proof}
    PLEASE FINISH THIS LATER.

\end{proof}

\begin{cor}
    If $\cis{\theta} = \cis{\phi}$, then $\theta \equiv \pi \pmod{2\pi}$.
\end{cor}

Use the fact that $\cos$ restricted to $[0,\pi]$ is continuous.

Now we will show $\lim_{\theta \to 0} \frac{\sin \theta}{\theta} = 1$.

\end{document}


