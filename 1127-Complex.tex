\documentclass[notes]{subfile}

\begin{document}
\section{Nov. 27, 2018}

A \term{conformal mapping} preserves angles between curves.
The \term{signed angle} between two curves at a point $p$ is 
\[ \measuredangle_p^{\pm}(\gamma, \delta) = 
{\Arg} (\overline{\gamma'(t_0)} \delta'(s_0)),\]
where $\gamma(t_0) = \delta(s_0) = p$.

\begin{definition}
    $f$ is conformal if $\measuredangle_p^{\pm}(\gamma, \delta) = \measuredangle_p^{\pm}(f \circ \gamma, f \circ \delta)$ 
    for all points $p$ in the domain of $f$ and all \term{smooth paths}\footnote{A smooth function, in this case, means
    $\gamma'(t)$ exists and is a continuous function of $t$.} $\gamma$
    and $\delta$ through $p$ and in the domain of $f$.
\end{definition}
 
\begin{theorem}
    This theorem has two parts.
    \begin{enumerate}
        \item If $f$ is analytic on a region and has nonvanishing
            derivative, then $f$ is conformal on that region.

        \item If $f$ is conformal on a region and 
            $\frac{\partial f}{\partial x}$ and 
            $\frac{\partial f}{\partial y}$ are continuous,
            then $f$ is analytic.
    \end{enumerate}
\end{theorem}

\begin{lemma}
    If $f : U \to C$ is analytic on $U$ and if $\gamma : [0,1]
    \to U$ is differentiable at $t_0 \in (0,1)$,
    \[ (f \circ \gamma)' (t_0) = (f' \circ \gamma) (t_0)
    \gamma'(t_0). \]
\end{lemma}

\textbf{please include proofs of these theorems eventually..}



\subsection{Complex Integration}
We start, of course, with a definition.

\begin{definition}
    If $\gamma: [0,1] \to \dom{f} \subseteq \C$ is a 
    \term{piecewise smooth}\footnote{$\gamma$ is piecewise 
    smooth if $\gamma$ is smooth on the interior of finitely
    many subintervals of $[0,1]$.}
    path and if $f$ is continuous on a region,
    then we define:
    \[ \int_{\gamma} f(z) \; dz = \int_0^1 f(\gamma(t))\gamma'(t) \;dt. \]

\end{definition}

So why does this integral exist?\\

Since $\gamma$ is continuous and so is $f$, we know $f(\gamma(t))$
is continuous.
We also know $\gamma'(t)$ is continuous so $f(\gamma(t))\gamma'(t)$
is continuous, so the integral exists.

These integrals $\int_{\gamma}$ are called \term{path integrals}.
When computing path integrals, we need to examine $\R$ to $\C$ 
integrals.  Here's another definition.

\begin{definition}[$\R$ to $\C$ integrals]
    If $g(t) = u(t) + iv(t)$, then we define
    \[ \int_a^b g(t) \; dt = \int_a^b u(t) \; dt + i
    \int_a^b v(t) \; dt. \]
    Note that if $g(t)$ is piecewise continuous on $[a,b]$,
    then its Riemann integral exists.
\end{definition}

\begin{exercise}
    Compute $\int_0^{1+i} z^2 \; dz$.
\end{exercise}

We also have other integrals that will come to use:
\begin{definition}
    The \term{arclength integral} of $f$ along a path $\gamma$ is
    defined to be:
    \[ \int_{\gamma} f(z) \; |dz| = \int_0^1 f(\gamma(t)) 
    |\gamma'(t)| \; dt. \]
\end{definition}

The \term{length} of a path $\gamma$ is then:
\[ L(\gamma) = \int_{\gamma} 1 \; |dz| = \int_0^1 |\gamma'(t)| \; dt\]
A finite length curve is called \textit{rectifiable}.

\subsection{Existence of Zigzag paths}
Complete this later.



\subsection{Properties of Integrals}
Take $c \in \C$.
Here are two to start off:
\begin{align}
    \int_{\gamma} cf(z) \; dz &= c \int_{\gamma} f(z) \; dz. \\
    \int_{\gamma} (f(z) + g(z)) \; dz &=
        \int_{\gamma} f(z) \; dz + \int_{\gamma} g(z) \; dz.
\end{align}

\begin{exercise}
    Prove these two properties.
    It might be helpful to write a function $F$ in terms of
    its real and imaginary part: $F(t) = u(t) + iv(t)$.
\end{exercise}

Now here is another theorem that will be super useful later.
\begin{theorem}[ML Theorem]
    \[ \left| \int_{\gamma} f(z) \; dz \right| \le
        \int_{\gamma} |f(z)| \; |dz| \le
    \max_{z \in \ran{\gamma}} |f(z)| \cdot L(\gamma). \]
\end{theorem}

Note that $\max_{z \in \ran{\gamma}} = \max_{t \in [0,1]} |f(\gamma(t))$ exists by the extreme value theorem. 

To prove the ML theorem, we first need the following lemma.
\begin{lemma}[Triangle Inequality for Integrals]
    For a complex valued $F$,
    \[ \left| \int_a^b F(t) \; dt \right| \le 
    \int_a^b |F(t)| \; dt. \]
\end{lemma}

Here is a proof of this lemma.
\begin{proof}
    Let $F(t) = u(t) + iv(t)$ and
    \[ c = \int_a^b F(t) \; dt. \]
    If $|c| = 0$, then the statement holds.
    Now let $c = |c| e^{i\theta}$.
    Then since $e^{-i\theta}$ is a constant:
    \[ |c| = e^{-i\theta} \int_a^b F(t) \; dt = \int_a^b 
    F(t)e^{-i\theta} \; dt. \]
    We also know
    \[ |c| = \real{c} = \real \int_a^b F(t)e^{-i\theta} \; dt
        = \int_a^b \real{F(t) e^{-i\theta}} \; dt
        \le \int_a^b |e^{-i\theta} F(t)| \; dt 
    = \int_a^b |F(t)| \; dt. \]
    Therefore,
    \[ \left| \int_a^b F(t) \; dt \right| \le 
    \int_a^b |F(t)| \; dt. \]

\end{proof}

Now to the proof of the ML theorem.
\begin{proof}
    This is just a string of integrals:
    \begin{align*}
        \left| \int_{\gamma} f(z) \; dz \right| &= 
        \left| \int_0^1 f(\gamma(t))\gamma'(t) \; dt \right| \\
        &\le \int_0^1 |f(\gamma(t))\gamma'(t)| \; dt 
        \tag{Triangle Inequality} \\
        &= \int_0^1 |f(\gamma(t))| |\gamma'(t)| \; dt \\
        &\le \int_0^1 \max_{0 \le s \le 1} |f(\gamma(s))| \; dt
        \cdot \int_0^1 |\gamma'(t)| \; dt \\
        &= \max_{0 \le s \le 1} |f(\gamma(s))| \cdot L(\gamma) \\
        &= \max_{z \in \ran{\gamma}} |f(z)| \cdot L(\gamma).
    \end{align*}
    This finishes the proof.
\end{proof}

\subsection{Reversal and Concatenation of Paths}

\begin{definition}
    Given a path $\gamma : [0,1] \to \C$, its \term{reversal}
    is a path $-[\gamma] : [0,1] \to \C$ such that 
    $-[\gamma](t) = \gamma(1-t)$.
\end{definition}
We also write $[\gamma] = \gamma$ sometimes.

\begin{theorem}
    \[ \int_{-[\gamma]} f(z) \; |dz| = \int_{[\gamma]} f(z) \; dz. \]
\end{theorem}

\begin{proof}
    This has something to do with \term{Duhamel sums}, but I wasn't
    here. 
    Need to grab other's notes.
\end{proof}

For concatenation of paths, we extend our old definition of
path integrals harmlessly.
A path integral of $f(z)$ over $\gamma : [0,T] \to \C$ is:
\[ \int_{\gamma} f(z) \; dz = \int_0^T f(\gamma(t)) \gamma'(t) \; dt. \]
Now we are ready for concatenation of paths.

\begin{definition}
    Let 
    \begin{align*}
        \gamma_1 : [0,N] \to \C, \\
        \gamma_2 : [0,M] \to \C.
    \end{align*}
    Then 
    \[ ([\gamma_1] + [\gamma_2])(t) := 
        \begin{cases}
            \gamma_1(t) & 0 \le t \le N, \\
            \gamma_2(t-N) & N < t \le N+M.
        \end{cases}
    \]
\end{definition}

Immediately with this definition comes a theorem.
\begin{theorem}
    \[ \int_{[\gamma_1] + [\gamma_2]} f(z) \; dz =
    \int_{\gamma_1} f(z) \; dz + \int_{\gamma_2} f(z) \; dz. \]
\end{theorem}

Figure out the proof yourself!


\end{document}
